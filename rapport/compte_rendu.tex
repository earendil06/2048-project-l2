\documentclass[11pt]{article}
\usepackage[francais]{babel}
\usepackage[utf8]{inputenc} 


\title{Introduction au langage C\\Projet 2048}

\author{PASTOR Florent}

\date{\today}

\begin{document}

\maketitle

\section{Introduction}
Ce projet a été réalisé au cours de l'UE Introduction au langage C. Il a pour but de recréer le jeu 2048 et est présent sous deux versions : une version console et une version graphique.

\section{Structure des données}
J'ai choisi de représenter une partie de jeu comme une structure contenant :
\begin{itemize}
	\item une matrice 4x4 d'entiers pour chaque case de la grille,
	\item un entier représentant le score,
	\item un booléen pour continuer ou non une partie gagnante.
\end{itemize}

\section{Compilation}

\subsection{Version console}

La version console de ce projet peut être compilée sous plusieurs OS.

\subsubsection{Linux (32/64bits)}
Sous linux, par cette simple ligne de commandes (dans le répertoire courant) :
\begin{verbatim}
gcc -Wall -ansi -pedantic 2048.c -o executable.out
\end{verbatim}

\subsubsection{Windows 8}



\subsection{Version graphique}

La version graphique de ce projet ne peut être compilée que sous un environnement Unix. Elle nécessite la librairie externe Gtk+ 3.x. disponible par cette ligne de commandes :
\begin{verbatim}
sudo apt-get install libgtk-3-dev
\end{verbatim}

Dans le répertoire du fichier à compiler, il suffit de taper cette ligne de commandes :
\begin{verbatim}
gcc `pkg-config --cflags --libs gtk+-3.0` -Wall -ansi 
-pedantic 2048.c -o executable.out
\end{verbatim}

\end{document}







mingw32-gcc.exe   -c "C:\Users\Florent\Desktop\Projet 2048\console\2048.c" -o "C:\Users\Florent\Desktop\Projet 2048\console\2048.o"
mingw32-g++.exe  -o "C:\Users\Florent\Desktop\Projet 2048\console\2048.exe" "C:\Users\Florent\Desktop\Projet 2048\console\2048.o"